\documentclass{article}\usepackage[]{graphicx}\usepackage[]{xcolor}
% maxwidth is the original width if it is less than linewidth
% otherwise use linewidth (to make sure the graphics do not exceed the margin)
\makeatletter
\def\maxwidth{ %
  \ifdim\Gin@nat@width>\linewidth
    \linewidth
  \else
    \Gin@nat@width
  \fi
}
\makeatother

\definecolor{fgcolor}{rgb}{0.345, 0.345, 0.345}
\newcommand{\hlnum}[1]{\textcolor[rgb]{0.686,0.059,0.569}{#1}}%
\newcommand{\hlsng}[1]{\textcolor[rgb]{0.192,0.494,0.8}{#1}}%
\newcommand{\hlcom}[1]{\textcolor[rgb]{0.678,0.584,0.686}{\textit{#1}}}%
\newcommand{\hlopt}[1]{\textcolor[rgb]{0,0,0}{#1}}%
\newcommand{\hldef}[1]{\textcolor[rgb]{0.345,0.345,0.345}{#1}}%
\newcommand{\hlkwa}[1]{\textcolor[rgb]{0.161,0.373,0.58}{\textbf{#1}}}%
\newcommand{\hlkwb}[1]{\textcolor[rgb]{0.69,0.353,0.396}{#1}}%
\newcommand{\hlkwc}[1]{\textcolor[rgb]{0.333,0.667,0.333}{#1}}%
\newcommand{\hlkwd}[1]{\textcolor[rgb]{0.737,0.353,0.396}{\textbf{#1}}}%
\let\hlipl\hlkwb

\usepackage{framed}
\makeatletter
\newenvironment{kframe}{%
 \def\at@end@of@kframe{}%
 \ifinner\ifhmode%
  \def\at@end@of@kframe{\end{minipage}}%
  \begin{minipage}{\columnwidth}%
 \fi\fi%
 \def\FrameCommand##1{\hskip\@totalleftmargin \hskip-\fboxsep
 \colorbox{shadecolor}{##1}\hskip-\fboxsep
     % There is no \\@totalrightmargin, so:
     \hskip-\linewidth \hskip-\@totalleftmargin \hskip\columnwidth}%
 \MakeFramed {\advance\hsize-\width
   \@totalleftmargin\z@ \linewidth\hsize
   \@setminipage}}%
 {\par\unskip\endMakeFramed%
 \at@end@of@kframe}
\makeatother

\definecolor{shadecolor}{rgb}{.97, .97, .97}
\definecolor{messagecolor}{rgb}{0, 0, 0}
\definecolor{warningcolor}{rgb}{1, 0, 1}
\definecolor{errorcolor}{rgb}{1, 0, 0}
\newenvironment{knitrout}{}{} % an empty environment to be redefined in TeX

\usepackage{alltt}
\usepackage[margin=1.0in]{geometry} % To set margins
\usepackage{amsmath}  % This allows me to use the align functionality.
                      % If you find yourself trying to replicate
                      % something you found online, ensure you're
                      % loading the necessary packages!
\usepackage{amsfonts} % Math font
\usepackage{fancyvrb}
\usepackage{hyperref} % For including hyperlinks
\usepackage[shortlabels]{enumitem}% For enumerated lists with labels specified
                                  % We had to run tlmgr_install("enumitem") in R
\usepackage{float}    % For telling R where to put a table/figure
\usepackage{natbib}        %For the bibliography
\bibliographystyle{apalike}%For the bibliography
\IfFileExists{upquote.sty}{\usepackage{upquote}}{}
\begin{document}
  \begin{enumerate}
  %%%%%%%%%%%%%%%%%%%%%%%%%%%%%%%%%%%%%%%%%%%%%%%%%%%%%%%%%%%%%%%%%%%%%%%%%%%%%%
  % Question 1
  %%%%%%%%%%%%%%%%%%%%%%%%%%%%%%%%%%%%%%%%%%%%%%%%%%%%%%%%%%%%%%%%%%%%%%%%%%%%%%
    \item Write a \texttt{pois.prob()} function that computes $P(X=x)$, 
    $P(X \neq x)$, $P(X<x)$, $P(X \leq x)$, $P(X > x)$, and $P(X \geq x).$ Enable the user to specify the rate parameter $\lambda$.
\begin{knitrout}\scriptsize
\definecolor{shadecolor}{rgb}{0.969, 0.969, 0.969}\color{fgcolor}\begin{kframe}
\begin{alltt}
\hldef{pois.prob} \hlkwb{<-} \hlkwa{function}\hldef{(}\hlkwc{x}\hldef{,} \hlkwc{lambda}\hldef{,} \hlkwc{type}\hldef{)\{}
  \hlkwa{if} \hldef{(type} \hlopt{==} \hlsng{"="}\hldef{)\{}
    \hlkwd{return} \hldef{(}\hlkwd{dpois}\hldef{(x, lambda))} \hlcom{#P(X=x) def of PMF}
  \hldef{\}} \hlkwa{else if} \hldef{(type}\hlopt{==} \hlsng{"!="}\hldef{) \{}
    \hlkwd{return} \hldef{(}\hlnum{1} \hlopt{-} \hlkwd{dpois}\hldef{(x, lambad))} \hlcom{# complement rule }
  \hldef{\}} \hlkwa{else if} \hldef{(type}\hlopt{==} \hlsng{"<"}\hldef{)\{}
    \hlkwd{return}\hldef{(}\hlkwd{ppois}\hldef{(x}\hlopt{-}\hlnum{1}\hldef{, lambda))}  \hlcom{# P(X < x) = P(X <= x-1)}
  \hldef{\}} \hlkwa{else if} \hldef{(type} \hlopt{==} \hlsng{"<="}\hldef{) \{}
    \hlkwd{return}\hldef{(}\hlkwd{ppois}\hldef{(x, lambda))} \hlcom{# P (X < = x) def of CDF}
  \hldef{\}} \hlkwa{else if} \hldef{(type} \hlopt{==} \hlsng{">"}\hldef{) \{}
    \hlkwd{return}\hldef{(}\hlnum{1} \hlopt{-} \hlkwd{ppois}\hldef{(x, lambda))} \hlcom{# complement rule}
  \hldef{\}} \hlkwa{else if} \hldef{(type} \hlopt{==} \hlsng{">="}\hldef{)\{}
    \hlkwd{return}\hldef{(}\hlnum{1} \hlopt{-} \hlkwd{ppois}\hldef{(x}\hlopt{-}\hlnum{1}\hldef{, lambda))} \hlcom{# P (X >= x) = 1 - P(X<x) = 1 - P(X <= x-1)}
  \hldef{\}}
\hldef{\}}
\end{alltt}
\end{kframe}
\end{knitrout}
In this step, I implemented \texttt{pois.prob()}, a function that computes 
probabilities from a Poisson distribution given an input value \texttt{x},
rate parameter \texttt{lambda}, and a probability type. The Poisson distribution
is a discrete probability distribution, meaning it assigns probabilities
to specific values of the random variable $X$. We calculate the probability
at a given value \texttt{x} using the probability mass function (PMF).


When the type is \texttt{"="}, the function computes the probability of
\texttt{X=x} using the PMF. For \texttt{"!="}, the probability is computed as 
the complement of \texttt{P(X=x)}, i.e., $P (X \neq x) = 1 - P(X=x)$. To compute
$P(X < x)$, we use $P(X \leq x-1)$, which is given by the CDF at $x-1$.
This works because the Poisson distribution only takes integer values. Therefore, since
the probability $X$ is strictly less than $x$, it is equivalent to the probability
that $X$ is less than or equal to $x-1$, so we can calculate this as the CDF
at $x-1$. For \texttt{"<="}, we directly use the CDF, as that is the definition.
Finally, we use the complement rule for \texttt{">"} and \texttt{">="}.


The key distinction here is that Poisson distributions are discrete, and thus
we can compute the probability of specific values of $X$. For probabilities 
involving inequalities, we utilize the CDF and apply the complement rule and the
fact that discrete distributions only take on integer values when needed.

  %%%%%%%%%%%%%%%%%%%%%%%%%%%%%%%%%%%%%%%%%%%%%%%%%%%%%%%%%%%%%%%%%%%%%%%%%%%%%%
  % Question 2
  %%%%%%%%%%%%%%%%%%%%%%%%%%%%%%%%%%%%%%%%%%%%%%%%%%%%%%%%%%%%%%%%%%%%%%%%%%%%%%
    \item Write a \texttt{beta.prob()} function that computes $P(X=x)$, 
    $P(X \neq x)$, $P(X<x)$, $P(X \leq x)$, $P(X > x)$, and $P(X \geq x)$
    for a beta distribution. Enable the user to specify the shape parameters
    $\alpha$ and $\beta$.
\begin{knitrout}\scriptsize
\definecolor{shadecolor}{rgb}{0.969, 0.969, 0.969}\color{fgcolor}\begin{kframe}
\begin{alltt}
\hldef{beta.prob} \hlkwb{<-} \hlkwa{function}\hldef{(}\hlkwc{x}\hldef{,} \hlkwc{alpha}\hldef{,} \hlkwc{beta}\hldef{,} \hlkwc{type}\hldef{)\{}
  \hlkwa{if} \hldef{(type} \hlopt{==} \hlsng{"="}\hldef{)\{}
    \hlkwd{return}\hldef{(}\hlnum{0}\hldef{)}                 \hlcom{# prob of exact value of continuous dist always 0}
  \hldef{\}} \hlkwa{else if} \hldef{(type}\hlopt{==} \hlsng{"!="}\hldef{) \{}
    \hlkwd{return}\hldef{(}\hlnum{1}\hldef{)}                 \hlcom{# complement rule}
  \hldef{\}} \hlkwa{else if} \hldef{(type}\hlopt{==} \hlsng{"<"}\hldef{)\{}
    \hlkwd{return}\hldef{(}\hlkwd{pbeta}\hldef{(x, alpha, beta))} \hlcom{# P(X<x) bc continous = P(X<=x)}
  \hldef{\}} \hlkwa{else if} \hldef{(type} \hlopt{==} \hlsng{"<="}\hldef{) \{}
    \hlkwd{return}\hldef{(}\hlkwd{pbeta}\hldef{(x, alpha, beta))} \hlcom{# P(X<=x) def of CDF}
  \hldef{\}} \hlkwa{else if} \hldef{(type} \hlopt{==} \hlsng{">"}\hldef{) \{}
    \hlkwd{return}\hldef{(}\hlnum{1} \hlopt{-} \hlkwd{pbeta}\hldef{(x, alpha, beta))} \hlcom{# complement rule}
  \hldef{\}} \hlkwa{else if} \hldef{(type} \hlopt{==} \hlsng{">="}\hldef{)\{}
    \hlkwd{return}\hldef{(}\hlnum{1} \hlopt{-} \hlkwd{pbeta}\hldef{(x, alpha, beta))} \hlcom{# same as above bc continuous}
  \hldef{\}}
\hldef{\}}
\end{alltt}
\end{kframe}
\end{knitrout}

In this step, I implemented \texttt{beta.prob()}, a function that computes
probabilities from a Beta distribution given an input value \texttt{x},
shape parameters \texttt{alpha} and \texttt{beta}, and a probability type.
Unlike the Poisson distribution, which is discrete, the Beta distribution is
a continuous probability, meaning it does not assign probabilities to
specific values of $X$. Instead, it assigns probabilities to intervals of $X$,
so the probability of any exact value is always zero. Therefore, when calculating 
probabilities, we never use the probability mass function (PMF). We only use the
cumulative distribution function (CDF) to calculate probabilities.


When the type is \texttt{"="}, the probability of \texttt{X=x} is zero,
as continuous distributions do not assign probabilities to single points.
For \texttt{"!="}, the probability is computed as the complement, meaning
$P(X \neq x) = 1$ because the probability of any specific value is always zero.
To compute $P(X < x)$ or $P(X \leq x)$, we use the CDF. In continuous distributions,
$P(X < x) = P(X \leq x)$ because the probability of a single point is zero.
Similarly, for \texttt{">"} and \texttt{">="}, these values are equal, and we 
again apply the complement rule.


In summary, the key difference between continuous and discrete distributions is
that for continuous distributions, we cannot compute probabilities of exact values
of $X$. Instead, probabilities are assigned to intervals, and we rely on the CDF
for all calculations.

\end{enumerate}
\bibliography{bibliography}
\end{document}
